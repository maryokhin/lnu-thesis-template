\chapter{Method}
\emph{Start with some text describing the content of the chapter.}\\

\noindent In this chapter you describe your scientific approach. That is how you plan to complete your work and how you want to reach your results, and the methods that you have chosen to validate the results. A method chapter can be divided into different sections. Below is a proposed subdivision. The Method chapter can also be added as Chapter 2, before the theory chapter.

In a research plan you describe how to proceed in your work, this is then written in future tense form. The Method chapter shall however be writing in the past tense. One can write a version of the method chapter before implementation and then update and change this after implementation. However, be sure that this update really is done!

In the method chapter, it is important to focus on the method you plan to use. However, you must show that you made a conscious and relevant choice of method. It can therefore be good to mention very briefly which methods are available as options. There must be a clear justification for the choice of method that has been done.

Different methods and techniques are described in the literature. The most important thing is that in the Method chapter you describe how you want to do and then take evidence in the literature to justify it. It can thus be good to very short reproduce what is in the literature, but it is important to remeber that the method chapter should not become a method textbook.

\section{Scientific approach}
In this section, you can initially describe if you have chosen an inductive or deductive approach. You can also describe if you will make a qualitative or quantitative study. If you have used an engineering approach you should descibe that. Of course, any choice should be justified.

\section{Data Collection}
Here you describe what data collection techniques you have used. You should mention briefly alternative and discuss why these techniques have not been selected. 

\subsection{Software Development}
When you  develop new software you need to describe the system and the requirements that should be fulfilled. It is not enough to have a system running you should also show that the requirements are fulfilled. The common approach to do this is:
\begin{itemize}
	\item Making relevant measurements
	\item Document studies
	\item Creating theoretical proofs
\end{itemize}

\subsection{Human centered approach}
For this approach the most common data collection techniques are:
\begin{itemize}
	\item Surveys
	\item Interviews
	\item Document studies
	\item Observations
\end{itemize}

\subsection{Selection}
When collecting data you have rarely been able to do it from all the people within the study area (population), which is called making a census survey. Usually, one needs to make any kind of selection. People who are asked in interviews and observed known informants and people who respond to surveys called respondents. There are different kinds of selections that can be made, such as convenience sampling, strategic sampling, quota sampling, representative sample. See methodological literature for detailed descriptions of these. In this section you describe the sample that has been applied and why.

\subsection{Implementation}
This section describes the implementation of data collection. You can describe your approach for designing questions for interviews or surveys. The actual interview questions and the questionnaires may well be added in an appendix, which is referred to here. You describe how the interviews were done or when questionnaires sent out and how, how interviews are documented and how questionnaires were collected. Most often, one must not reply to any inquiries that you sent out. Here you can also specify and discuss the response rate that you had. It is important to be clear about how you proceeded, but do not go to deeply into details. It is also important to be honest and not avoid to mention problems, for example, the loss of respondents.

\section{Analysis}
Here you describe the techniques you have used to analyze the results. Examples of quantitative analyzes are different statistical tests. Examples of qualitative analysis are content analysis and coding. You use the theory in the analysis to describe and explain the result. For example, a used theoretical model, which describes what you have studied. Although you did not use a particular technology, it is important to describe how you have proceeded in the analysis. This is something that both report writers and researchers miss.

\section{Reliability}
An report must be of a high scientific quality. The knowledge that you produce must be trustworthy and reliable. Two concepts are used to discuss the scientific quality - reliability and validity. In this section you can discuss the reliability and validity of the study you have done. This discussion can also be added to the Discussion chapter.

\section{Ethical considerations}
%This section is not mandatory but may be appropriate to add in some types of surveys.
Studies in which ethical considerations should be taken into account are those that infringe on people's privacy. Examples of studies where ethical considerations should be done is participant observation in the workplace, deep personal interviews as well as interviews and surveys with issues affecting people's privacy.

Investigations related to patients in health care must always undergo ethical review to be carried out. It also includes studies on IT systems and information management in healthcare. Such examination conducted by the Ethics Committee of the South-East, \cite{etik}, but first you should make their self-assessment.
