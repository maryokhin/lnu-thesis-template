\chapter{Conclusion}
\emph{Start with some text describing the content of the chapter.}\\

\noindent The report ends with a conclusion and finally suggestions for further research. This can be written in a separate chapter or at the end of the discussion. The finish can be read independently and it is thus preferable to begin with a very brief statement of what you have done and repeat the purpose and / or the issues. Nothing new is entered in the finish and normally no need for ongoing reference.

\section{Conclusions}
The conclusion should answer the questions in a very concise way. Alternatively you reproduces if you have been able to verify or falsify your hypothesis and what arguments you have for this. The conclusion must be very short and concise, preferably no more than one paragraph long and should not contain any figures.

\section{Further research}
Here you give suggestions for further research related to your research area and to your findings. There may be things you have not had time or been able to do. Here you can look back into the section Limitations. Possibly, that section also will be updated along with suggestions for further research. Another section to look at is the discussion of validity. There may be a need of more studies to really confirm the results. These proposals can be the basis for your own further research on the next level or for other students to write a new report. Within the research area we are also supposed to work cumulative, which means that we deepen and develop what others have done before.
