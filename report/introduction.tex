\chapter{Introduction}
\emph{Start with some text describing the content of the chapter.}\\

\noindent This chapter will describe relevant background information for the thesis work and can be divided into different sub-chapters. Below is a suggestion, but what you use depends on what type of report you are writing and what sort of research you do. It is also possible to have the headlines in a different order if you so wish.

The introduction should pique the reader's interest in the paper and reproduce enough background information for the reader to understand the problem statement. The introduction should not be too long, then it is easy to be already here loses the reader's interest. Therefore, it shall only contain such descriptions that are relevant.

Introduction printed with a mixture of the present, past and future. For example, the present tense for what you and others think and how different things relate and the themes of the moment of writing. Imperfect of what other researchers have done and concluded. Future tense for what the investigation intends to do and what the report will describe.

One can write a first version of its introduction as the first step in the thesis. During the work, you might change this a bit. When the report is getting ready you go back and adjust the final version. It happens that the aim and questions changed during the work. It can also happens that new previous research will be added. Finally, you also want to hone some more on his introduction when you have the results, discussion and conclusion clear, so that all these parts become consistent.

\section{Introduction / Background}
Describe very briefly and in general terms what the project will be about. Describe maybe how you came to write about this. That one slips into a specific area may be due to initially have a query. This question is not as precise as a research question, but will perhaps lead to one. It should also explain why the topic chosen is interesting and relevant. A good way might be to pick up something from a paper, the media or the public debate to show that the topic is relevant.

Some studies involving some form of practical problem solving, perhaps on behalf of any company. Describe the practical problem area here. It may mean a brief description of the activities and mission.

In this and the next section, one can conveniently make use of a so-called "funnel" technique when writing. It involves describing the area very wide to begin with, and then taper off more and more until you get down to the report question.

\section{Previous research}
Here you describe briefly what others have done in the field or how others have attempted to explain or solve the problem that you intend to study. This section shall contain only brief and relevant commentary, but you should connect it to your coming work. The literature they were mainly based on the essays, scholarly articles and books. Reference to the references that you have used will be made subsequently. Look into the theory section of this document how these references are supposed to be made.

The section may also contain description of the theoretical background needed to describe the problem. Where do you do most often a summary here and then develop descriptions more in the theory chapter.

\section{Problem definition}
Here you describe the problem that you intend to investigate. In order for work to be counted as an report you must investigate a scientific problem. In the research conducted in computer science, there is the possibility of also working with a practical problem, but this should still be linked to a scientific problem. If so, describe it then first the practical problem, and then the scientific problem. One can imagine the problem as a kind of knowledge gap that you want to make a contribution to by adding new knowledge. Examples of the problem may be that you do not know what people do on a particular issue, you do not know how something works, you do not know the reason to your problem or the lack of methods.

\section{Purpose and research question / hypothesis}
The purpose describes what you intend to do to investigate the problem and fill the knowledge gap. You could say that it is a kind of synthesis of the implementation. The purpose should include a specification of the type of knowledge they intend to produce. For example, descriptive knowledge, explanatory knowledge and normative knowledge (which means methods).

Depending on whether you are working deductively or inductively you should clarify its purpose in one or more hypotheses or questions. A hypothesis is an assumption about how something relates, that it intends to investigate whether it is true or not. For example, to assume that all systems development companies follow a system development method, which it intends to investigate whether it is true or not. The hypothesis is always based in theory or prior research. We must therefore know something before we can formulate a hypothesis. This means working deductively.

If you do not know something (not you personally, but the scientific community), you get instead work inductively. Then we formulate one or more research questions. These questions must be answered by the inquiry to make. Conclusion of the report is thus a direct response to the research questions. A research question can be for example; "What systems development methodologies are used in systems development company?". The research questions are clearer specification of the initial puzzlement as they might have.

\section{Scope / Limitation}
Here you describe what you do not intend to do in your investigation. It is important not to confuse this with the selection describing the methodology chapter. Therefore, you should not write that you will interview the X and Y, but not Z. Your limitations should be relevant to the purpose in question. Therefore, they should not specify that they opted out to study the rest of the world. Some limitations make you maybe from the beginning, while others naturally to come during the work. Explain also why this will not be studied. Time is a common reason why aspects must be deselected. Note, however, to describe it in a professional manner and you shall not make excuses.

\section{Target group}
This section is not mandatory. However, this can be good both when writing and later on for the reader. Here you describe for whom you intend to give a knowledge contribution to. If I as a reader is mentioned in the target group, I know of course that this paper should be of interest for me to read.

If you plan to make a practical contribution to a specific group of persons or activities you describe it here. It is not enough to mention this target group to classify this report as a scientific paper. You must target it to a more general and scientific audience. To describe this audience here and constantly thinking about writing for them is a great way to ensure that it becomes a scientific paper writing.

It is not enough to have the supervisor and examiner as that audience. If you can not come to a broader audience, perhaps the  chosen topic is not interesting, relevant and innovative enough.

\section{Outline}
For the reader it can be helpful to have a description of the various coming chapters and sections of the report and how these relate to each other. Here we can also provide reading instructions for specific groups of readers if you so wish.
