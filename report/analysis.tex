\chapter{Analysis}
\emph{Start with some text describing the content of the chapter.}\\

\noindent In this chapter you describe your analysis. How this is done depends on the type of analysis to do. In a quantitative study, the results of various statistical analyzes and comparisons of different issues are done here. It is also possible to aggregate the result and analysis chapter to one chapter. In a qualitative study, one can demonstrate the patterns that have been  recognized based on the material. It may also be in order to apply a theoretical model on the empirical material, or otherwise make connections between theory and empirical data to describe, explain or point out the connection. The analysis shall be more clean and maintained without discussion. In qualitative studies, it is common to add the analysis and discussion chapter to one chapter.
