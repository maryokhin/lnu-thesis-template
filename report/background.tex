\chapter{Background / Theory}
\emph{Start with some text describing the content of the chapter.}\\

\noindent Here you describe the thesis background and or theory. The chapter or chapters divided into electives relevant sections. It is possible to have both a background chapter and a theory chapter if you think this is needed. Theory chapter may be number 2, but it is also possible to add the method chapter before theory chapter. It depends on whether you think the reader needs to know the theories in order to understand the approach of the survey.
		
\section{Introduction}
Theories describes and explains the area and various phenomena and their relationship. The theory is used to analyze, i.e., to describe and explain what you are studying, or as a basis to formulate a hypothesis. Theories most often are described in books, but can also be described in scientific articles.

The theories shall be summarized and presented briefly with references to the literature. Describe the theories in your own words. Quotes may be used , but should be done sparingly. Maybe not the whole theory will be used , describe only the part of the theory used. Only relevant, that is, the theories that you will actually use, should be described. However, you should show that you made a conscious choice of theories. Thus one can write a few short sentences that these theories exist, but these theories have been chosen because $\ldots$ and so on. Details about referring to literature, quotes and plagiarism is given in \cite{refero08}. % No entry in bibliography provided by LNU here.

In the start you very often write a lot in this chapter, but later on some of the parts will be deleted. Make sure to really "Kill your darlings ", i.e., just think if it really fit! You should also describe how the theory will be used in your analysis. If you can not describe this, it is quite likely that the theory is also not useful and therefore should be removed. If you describe several theories you should also make a clear link between the theories.

In this chapter, you can develop descriptions of previous research that did not fit in the introduction. If you make a practical study, you can use  this chapter for a more detailed description of the practical environment. It is also in a background chapter that you describe theory that is needed to describe the problem, but that will not be used in the analysis. Examples of this include the definitions of the various concepts in the field.
