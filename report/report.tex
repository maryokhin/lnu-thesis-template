%----------------------------------------------------------------------------------------
%	Setup and document configuration
%----------------------------------------------------------------------------------------
\documentclass[a4paper, 12pt, titlepage, abstract, numbers=enddot]{scrreprt}

\usepackage[T1]{fontenc} % 8-bit encoding which has 256 glyphs
\usepackage{mathtools} % Support for advanced formulas
\usepackage{accents} % Support for advanced accents
\usepackage{mathptmx} % Use Times as default text font, and provide maths support
\usepackage[utf8]{inputenc}
\usepackage{dtklogos} % Logos for i.e. LaTeX, BibTeX, etc.
\usepackage{wallpaper} % Build background
\usepackage[absolute]{textpos} % Ability to absolutely position the text
\usepackage{subcaption} % Ability to position figures side by side
\usepackage[top=2cm, bottom=2.5cm, left=3cm, right=3cm]{geometry} % Set margins
\usepackage{appendix} % Support for separate handling of attachments
\usepackage[linktoc=all]{hyperref} % Support hyper-linking of Table of Contents
\usepackage[figure]{hypcap} % Point the hyperlink to the top of figures and tables
\usepackage[noabbrev]{cleveref} % Reference macros depending on type of referenced object
\usepackage{csquotes} % Handles the quotes
\usepackage[backend=biber, style=ieee, sorting=none]{biblatex} % Handles the bibliography

\setkomafont{disposition}{\bfseries} % Switch to serif for headings
\addtokomafont{chapter}{\fontsize{14}{15}\selectfont}
\addtokomafont{section}{\fontsize{12}{15}\selectfont}
\addtokomafont{subsection}{\fontsize{12}{15}\selectfont}

\setcounter{secnumdepth}{3} % Three levels of heading numbering
\setcounter{tocdepth}{3} % Three levels of heading numbering in the table of contents

% Partial compilation possible when working on specific chapters
\includeonly{introduction,background,method,results,analysis,discussion, conclusion} 

\addbibresource{refs.bib} % Bibliography source 


\begin{document}
\pagenumbering{gobble}

\newgeometry{top=0cm, left=5cm}

%----------------------------------------------------------------------------------------
%	Tassel environment configuration
%----------------------------------------------------------------------------------------
\newsavebox{\mybox}
\newlength{\mydepth}
\newlength{\myheight}

\newenvironment{tassel}
{\begin{lrbox}{\mybox}\begin{minipage}{\textwidth}}
{\end{minipage}\end{lrbox}
 \settodepth{\mydepth}{\usebox{\mybox}}
 \settoheight{\myheight}{\usebox{\mybox}}
 \addtolength{\myheight}{\mydepth}
 \noindent\makebox[0pt]{\hspace{-20pt} % inner offset
 \rule[-\mydepth]{1pt}{\myheight}} % tassel width
 \usebox{\mybox}}
%----------------------------------------------------------------------------------------

\begin{titlepage}

\setlength{\wpXoffset}{-8.3cm}
\setlength{\wpYoffset}{12.5cm}
\ThisCenterWallPaper{0.1}{../img/lnu/logo.png}

\ThisLLCornerWallPaper{0.45}{../img/lnu/branch.png}

\begin{tassel}
    \vspace{7.5cm} % tassel length
    \normalfont \normalsize
    \huge Type of document \\ % Document type
    \vspace{-1.3cm}
\end{tassel}

\vspace{3cm}

\begin{flushleft}
    \normalfont 
    \huge Title \\ % Document title
    \huge \it Degree project in progress  % Optional document subheading
\end{flushleft}

\null
\vfill

\begin{textblock}{6}(10,13)
\begin{flushright}
\begin{minipage}{\textwidth}
\begin{flushleft} \normalfont \large
\emph{Author:} John \textsc{Smith}\\
\emph{Supervisor:} Dr.~Foo \textsc{Bar}\\
\emph{Examiner:} Dr.~Mark \textsc{Brown}\\
\emph{Semester:} HT2014\\
\emph{Subject:} Any Science\\
\emph{Course code:} xDVxxE
\end{flushleft}
\end{minipage}
\end{flushright}
\end{textblock}

\end{titlepage}

\restoregeometry

 
\begin{abstract}
The report can have a summary or an abstract. An abstract is a more scientific way to
summarize the report. The abstract should be no longer than a paragraph, and is not
divided into more than one piece. In the abstract, you are answering the questions:

\begin{itemize}
	\item In what context is the problem raised?
	\item What was the problem? Why is it iteresting and important to study this specific problem?
	\item What has been done to solve the problem?
	\item What is the result?
\end{itemize}

Sometimes the abstract is followed by some keywords, which are representative of the thesis area.

From reading the summary or the abstract the reader should clearly understand what the report is all about. It is often said that the title of the report then is a quick summary. That is the title of the report should also give a clear indication of what the report is about. If you have keywords is a good idea that the title contains a large proportion of these. Then you know you have an accurate title. In our data, scientific reports, we use the format of the abstract.\\

\textbf{Keywords:}
Thesis, report, template.

\end{abstract}
%----------------------------------------------------------------------------------------
%	Preface
%----------------------------------------------------------------------------------------
\chapter*{Preface}
You can have a preface in the paper if you want, but it is not necessary. In this you can write more personal reflections on report writing. An example would be why you yourself personally thought it was an interesting topic to write about or if you have experienced it particularly difficult or rewarding to write the report.

It is also in the preface you can take this opportunity to thank the people who have been particularly helpful during the report writing, for example, if you had any contact with a company that helped, those who have participated in interviews or questionnaires, if you received any funding from any or if you had someone who helped with proofreading.

The preface should not be long-winded, maximum half a page. Also note anonymity and confidentiality before you mention people by name. You can thank the people anyway, without typing their name.
%----------------------------------------------------------------------------------------
%   Table of contents, list of figures
%----------------------------------------------------------------------------------------
\pagenumbering{gobble} % Turn off page numbering after the table of contents
\tableofcontents
\listoffigures
\newpage
\pagenumbering{arabic} % Start page numbering at 1

%----------------------------------------------------------------------------------------
%    Chapters
%   Each chapter is a separate *.tex document
%   Selective compilation of chapters by setting \includeonly on the top
%----------------------------------------------------------------------------------------
\chapter{Introduction}
\emph{Start with some text describing the content of the chapter.}\\

\noindent This chapter will describe relevant background information for the thesis work and can be divided into different sub-chapters. Below is a suggestion, but what you use depends on what type of report you are writing and what sort of research you do. It is also possible to have the headlines in a different order if you so wish.

The introduction should pique the reader's interest in the paper and reproduce enough background information for the reader to understand the problem statement. The introduction should not be too long, then it is easy to be already here loses the reader's interest. Therefore, it shall only contain such descriptions that are relevant.

Introduction printed with a mixture of the present, past and future. For example, the present tense for what you and others think and how different things relate and the themes of the moment of writing. Imperfect of what other researchers have done and concluded. Future tense for what the investigation intends to do and what the report will describe.

One can write a first version of its introduction as the first step in the thesis. During the work, you might change this a bit. When the report is getting ready you go back and adjust the final version. It happens that the aim and questions changed during the work. It can also happens that new previous research will be added. Finally, you also want to hone some more on his introduction when you have the results, discussion and conclusion clear, so that all these parts become consistent.

\section{Introduction / Background}
Describe very briefly and in general terms what the project will be about. Describe maybe how you came to write about this. That one slips into a specific area may be due to initially have a query. This question is not as precise as a research question, but will perhaps lead to one. It should also explain why the topic chosen is interesting and relevant. A good way might be to pick up something from a paper, the media or the public debate to show that the topic is relevant.

Some studies involving some form of practical problem solving, perhaps on behalf of any company. Describe the practical problem area here. It may mean a brief description of the activities and mission.

In this and the next section, one can conveniently make use of a so-called "funnel" technique when writing. It involves describing the area very wide to begin with, and then taper off more and more until you get down to the report question.

\section{Previous research}
Here you describe briefly what others have done in the field or how others have attempted to explain or solve the problem that you intend to study. This section shall contain only brief and relevant commentary, but you should connect it to your coming work. The literature they were mainly based on the essays, scholarly articles and books. Reference to the references that you have used will be made subsequently. Look into the theory section of this document how these references are supposed to be made.

The section may also contain description of the theoretical background needed to describe the problem. Where do you do most often a summary here and then develop descriptions more in the theory chapter.

\section{Problem definition}
Here you describe the problem that you intend to investigate. In order for work to be counted as an report you must investigate a scientific problem. In the research conducted in computer science, there is the possibility of also working with a practical problem, but this should still be linked to a scientific problem. If so, describe it then first the practical problem, and then the scientific problem. One can imagine the problem as a kind of knowledge gap that you want to make a contribution to by adding new knowledge. Examples of the problem may be that you do not know what people do on a particular issue, you do not know how something works, you do not know the reason to your problem or the lack of methods.

\section{Purpose and research question / hypothesis}
The purpose describes what you intend to do to investigate the problem and fill the knowledge gap. You could say that it is a kind of synthesis of the implementation. The purpose should include a specification of the type of knowledge they intend to produce. For example, descriptive knowledge, explanatory knowledge and normative knowledge (which means methods).

Depending on whether you are working deductively or inductively you should clarify its purpose in one or more hypotheses or questions. A hypothesis is an assumption about how something relates, that it intends to investigate whether it is true or not. For example, to assume that all systems development companies follow a system development method, which it intends to investigate whether it is true or not. The hypothesis is always based in theory or prior research. We must therefore know something before we can formulate a hypothesis. This means working deductively.

If you do not know something (not you personally, but the scientific community), you get instead work inductively. Then we formulate one or more research questions. These questions must be answered by the inquiry to make. Conclusion of the report is thus a direct response to the research questions. A research question can be for example; "What systems development methodologies are used in systems development company?". The research questions are clearer specification of the initial puzzlement as they might have.

\section{Scope / Limitation}
Here you describe what you do not intend to do in your investigation. It is important not to confuse this with the selection describing the methodology chapter. Therefore, you should not write that you will interview the X and Y, but not Z. Your limitations should be relevant to the purpose in question. Therefore, they should not specify that they opted out to study the rest of the world. Some limitations make you maybe from the beginning, while others naturally to come during the work. Explain also why this will not be studied. Time is a common reason why aspects must be deselected. Note, however, to describe it in a professional manner and you shall not make excuses.

\section{Target group}
This section is not mandatory. However, this can be good both when writing and later on for the reader. Here you describe for whom you intend to give a knowledge contribution to. If I as a reader is mentioned in the target group, I know of course that this paper should be of interest for me to read.

If you plan to make a practical contribution to a specific group of persons or activities you describe it here. It is not enough to mention this target group to classify this report as a scientific paper. You must target it to a more general and scientific audience. To describe this audience here and constantly thinking about writing for them is a great way to ensure that it becomes a scientific paper writing.

It is not enough to have the supervisor and examiner as that audience. If you can not come to a broader audience, perhaps the  chosen topic is not interesting, relevant and innovative enough.

\section{Outline}
For the reader it can be helpful to have a description of the various coming chapters and sections of the report and how these relate to each other. Here we can also provide reading instructions for specific groups of readers if you so wish.

\chapter{Background / Theory}
\emph{Start with some text describing the content of the chapter.}\\

\noindent Here you describe the thesis background and or theory. The chapter or chapters divided into electives relevant sections. It is possible to have both a background chapter and a theory chapter if you think this is needed. Theory chapter may be number 2, but it is also possible to add the method chapter before theory chapter. It depends on whether you think the reader needs to know the theories in order to understand the approach of the survey.
		
\section{Introduction}
Theories describes and explains the area and various phenomena and their relationship. The theory is used to analyze, i.e., to describe and explain what you are studying, or as a basis to formulate a hypothesis. Theories most often are described in books, but can also be described in scientific articles.

The theories shall be summarized and presented briefly with references to the literature. Describe the theories in your own words. Quotes may be used , but should be done sparingly. Maybe not the whole theory will be used , describe only the part of the theory used. Only relevant, that is, the theories that you will actually use, should be described. However, you should show that you made a conscious choice of theories. Thus one can write a few short sentences that these theories exist, but these theories have been chosen because $\ldots$ and so on. Details about referring to literature, quotes and plagiarism is given in \cite{refero08}. % No entry in bibliography provided by LNU here.

In the start you very often write a lot in this chapter, but later on some of the parts will be deleted. Make sure to really "Kill your darlings ", i.e., just think if it really fit! You should also describe how the theory will be used in your analysis. If you can not describe this, it is quite likely that the theory is also not useful and therefore should be removed. If you describe several theories you should also make a clear link between the theories.

In this chapter, you can develop descriptions of previous research that did not fit in the introduction. If you make a practical study, you can use  this chapter for a more detailed description of the practical environment. It is also in a background chapter that you describe theory that is needed to describe the problem, but that will not be used in the analysis. Examples of this include the definitions of the various concepts in the field.

\chapter{Method}
\emph{Start with some text describing the content of the chapter.}\\

\noindent In this chapter you describe your scientific approach. That is how you plan to complete your work and how you want to reach your results, and the methods that you have chosen to validate the results. A method chapter can be divided into different sections. Below is a proposed subdivision. The Method chapter can also be added as Chapter 2, before the theory chapter.

In a research plan you describe how to proceed in your work, this is then written in future tense form. The Method chapter shall however be writing in the past tense. One can write a version of the method chapter before implementation and then update and change this after implementation. However, be sure that this update really is done!

In the method chapter, it is important to focus on the method you plan to use. However, you must show that you made a conscious and relevant choice of method. It can therefore be good to mention very briefly which methods are available as options. There must be a clear justification for the choice of method that has been done.

Different methods and techniques are described in the literature. The most important thing is that in the Method chapter you describe how you want to do and then take evidence in the literature to justify it. It can thus be good to very short reproduce what is in the literature, but it is important to remeber that the method chapter should not become a method textbook.

\section{Scientific approach}
In this section, you can initially describe if you have chosen an inductive or deductive approach. You can also describe if you will make a qualitative or quantitative study. If you have used an engineering approach you should descibe that. Of course, any choice should be justified.

\section{Data Collection}
Here you describe what data collection techniques you have used. You should mention briefly alternative and discuss why these techniques have not been selected. 

\subsection{Software Development}
When you  develop new software you need to describe the system and the requirements that should be fulfilled. It is not enough to have a system running you should also show that the requirements are fulfilled. The common approach to do this is:
\begin{itemize}
	\item Making relevant measurements
	\item Document studies
	\item Creating theoretical proofs
\end{itemize}

\subsection{Human centered approach}
For this approach the most common data collection techniques are:
\begin{itemize}
	\item Surveys
	\item Interviews
	\item Document studies
	\item Observations
\end{itemize}

\subsection{Selection}
When collecting data you have rarely been able to do it from all the people within the study area (population), which is called making a census survey. Usually, one needs to make any kind of selection. People who are asked in interviews and observed known informants and people who respond to surveys called respondents. There are different kinds of selections that can be made, such as convenience sampling, strategic sampling, quota sampling, representative sample. See methodological literature for detailed descriptions of these. In this section you describe the sample that has been applied and why.

\subsection{Implementation}
This section describes the implementation of data collection. You can describe your approach for designing questions for interviews or surveys. The actual interview questions and the questionnaires may well be added in an appendix, which is referred to here. You describe how the interviews were done or when questionnaires sent out and how, how interviews are documented and how questionnaires were collected. Most often, one must not reply to any inquiries that you sent out. Here you can also specify and discuss the response rate that you had. It is important to be clear about how you proceeded, but do not go to deeply into details. It is also important to be honest and not avoid to mention problems, for example, the loss of respondents.

\section{Analysis}
Here you describe the techniques you have used to analyze the results. Examples of quantitative analyzes are different statistical tests. Examples of qualitative analysis are content analysis and coding. You use the theory in the analysis to describe and explain the result. For example, a used theoretical model, which describes what you have studied. Although you did not use a particular technology, it is important to describe how you have proceeded in the analysis. This is something that both report writers and researchers miss.

\section{Reliability}
An report must be of a high scientific quality. The knowledge that you produce must be trustworthy and reliable. Two concepts are used to discuss the scientific quality - reliability and validity. In this section you can discuss the reliability and validity of the study you have done. This discussion can also be added to the Discussion chapter.

\section{Ethical considerations}
%This section is not mandatory but may be appropriate to add in some types of surveys.
Studies in which ethical considerations should be taken into account are those that infringe on people's privacy. Examples of studies where ethical considerations should be done is participant observation in the workplace, deep personal interviews as well as interviews and surveys with issues affecting people's privacy.

Investigations related to patients in health care must always undergo ethical review to be carried out. It also includes studies on IT systems and information management in healthcare. Such examination conducted by the Ethics Committee of the South-East, \cite{etik}, but first you should make their self-assessment.

\chapter{Results / Empirical data}
\emph{Start with some text describing the content of the chapter.}\\

\noindent What you describe in this chapter depends very much on what type of study you have done. The division into sub-headings based consequently on this. In this part, the plain clean results shall be described without analyzes and discussions. It is a kind of summary of the data that you have collected. One should make a straight and more or less objective description. Only relevant material should be included. The chapter is written preferably in the present tense.
		
\section{Approaches for reporting results}
If you conducted a survey, you present the answers question by question. This can preferable be done in tables and charts with commentary text added to it. One should not present the same data in both table and graph, but choose the one that fits best. One does not render the entire contents of a table or chart in the text, but you should comment in some way. All tables and figures must also be mentioned in the text.

If you have done interviews you can here reproduce some sort of summary or compilation. you can feel free to bring specific quotes from interviews, but you should never reproduce the entire interview, not even in appendices. If you have done experiments you present the different outcome here. Have you done case studies of activities or processes you describe these here.

\section{More results}	
There may be a need to develop and clarify the work and the results obtained.

\chapter{Analysis}
\emph{Start with some text describing the content of the chapter.}\\

\noindent In this chapter you describe your analysis. How this is done depends on the type of analysis to do. In a quantitative study, the results of various statistical analyzes and comparisons of different issues are done here. It is also possible to aggregate the result and analysis chapter to one chapter. In a qualitative study, one can demonstrate the patterns that have been  recognized based on the material. It may also be in order to apply a theoretical model on the empirical material, or otherwise make connections between theory and empirical data to describe, explain or point out the connection. The analysis shall be more clean and maintained without discussion. In qualitative studies, it is common to add the analysis and discussion chapter to one chapter.

\include{discussion}
\chapter{Conclusion}
\emph{Start with some text describing the content of the chapter.}\\

\noindent The report ends with a conclusion and finally suggestions for further research. This can be written in a separate chapter or at the end of the discussion. The finish can be read independently and it is thus preferable to begin with a very brief statement of what you have done and repeat the purpose and / or the issues. Nothing new is entered in the finish and normally no need for ongoing reference.

\section{Conclusions}
The conclusion should answer the questions in a very concise way. Alternatively you reproduces if you have been able to verify or falsify your hypothesis and what arguments you have for this. The conclusion must be very short and concise, preferably no more than one paragraph long and should not contain any figures.

\section{Further research}
Here you give suggestions for further research related to your research area and to your findings. There may be things you have not had time or been able to do. Here you can look back into the section Limitations. Possibly, that section also will be updated along with suggestions for further research. Another section to look at is the discussion of validity. There may be a need of more studies to really confirm the results. These proposals can be the basis for your own further research on the next level or for other students to write a new report. Within the research area we are also supposed to work cumulative, which means that we deepen and develop what others have done before.


%----------------------------------------------------------------------------------------
%	Bibliography
%  IEEE system with references to figures and no sorting
%  Change the bibliography style for another reference system
%----------------------------------------------------------------------------------------
\printbibliography

%----------------------------------------------------------------------------------------
%	Attachments are managed in a separate file called the appendix
%-----------------------------------------------------------------------------------------
\clearpage
\setcounter{page}{1} % Starting on page 1 for attachments
\appendix

\chapter{Appendix 1} % Append '*' to skip from table of contents (TOC)
It may often be appropriate to put details in the appendices, as they briefly has been explained in the main report and there been given a reference to details via see Appendix X.

\begin{subappendices}

\section{Appendix 1.1 - Sub appendix}
It is possible to create appendices belonging to a common appendix, if that is needed.

\section{Appendix 1.2 - Another sub appendix}
More belonging to the same main theme.

\end{subappendices}

\chapter{Appendix 2}
Unless the content is linked you should create different attachments. Important to remember that all appendices must be referred to in the text of the report otherwise they should not be included.


\end{document}
