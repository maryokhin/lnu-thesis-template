\chapter{Results / Empirical data}
\emph{Start with some text describing the content of the chapter.}\\

\noindent What you describe in this chapter depends very much on what type of study you have done. The division into sub-headings based consequently on this. In this part, the plain clean results shall be described without analyzes and discussions. It is a kind of summary of the data that you have collected. One should make a straight and more or less objective description. Only relevant material should be included. The chapter is written preferably in the present tense.
		
\section{Approaches for reporting results}
If you conducted a survey, you present the answers question by question. This can preferable be done in tables and charts with commentary text added to it. One should not present the same data in both table and graph, but choose the one that fits best. One does not render the entire contents of a table or chart in the text, but you should comment in some way. All tables and figures must also be mentioned in the text.

If you have done interviews you can here reproduce some sort of summary or compilation. you can feel free to bring specific quotes from interviews, but you should never reproduce the entire interview, not even in appendices. If you have done experiments you present the different outcome here. Have you done case studies of activities or processes you describe these here.

\section{More results}	
There may be a need to develop and clarify the work and the results obtained.
